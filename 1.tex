\documentclass{article}


\usepackage{graphicx}
\usepackage{graphicx} % includegraphics command is implemented here
\usepackage{titling}
\usepackage{listings}
\usepackage{color}
\usepackage[font=scriptsize]{caption}

\definecolor{dkgreen}{rgb}{0,0.6,0}
\definecolor{gray}{rgb}{0.5,0.5,0.5}
\definecolor{mauve}{rgb}{0.58,0,0.82}

\lstset{frame=tb,
  language=Python,
  aboveskip=3mm,
  belowskip=3mm,
  showstringspaces=false,
  columns=flexible,
  basicstyle={\small\ttfamily},
  numbers=none,
  numberstyle=\tiny\color{gray},
  keywordstyle=\color{blue},
  commentstyle=\color{dkgreen},
  stringstyle=\color{mauve},
  breaklines=true,
  breakatwhitespace=true
  tabsize=3
}
\newcommand{\subtitle}[1]{%
  \posttitle{%
    \par\end{center}
    \begin{center}\large#1\end{center}
    \vskip0.2em}%
}
%pre-amble
\title{\textbf{Group Project}}
\members{ \emph{members:} Hao Chen, Hon Leung Lee, Xin Guo, Di Wu}
\date{May 21th, 2013}
\begin{document}
\maketitle % typeset the title
\section{Introduction}
     Topic 1: Investigate this challenge problem: 
     given a binary integer a, define an operation to be: add a any number of + signs between any two digits of a, 
     and compute the sum. It's clear given any number, finitely many operations will change this number a to 1. 
     Now the question is: How many operations do you need at most? An example: 
     a = 100111011 -> 100+1+1+10+11 = 1011 -> 10+1+1 = 100 -> 1+0+0 = 1 
     Goal: give an absolute constant C, such that for any number a, there exists a sequence of at most C operations, 
     changing a to 1.
     
     Topic 2: Find out the best algorithm to solve the famous Fermat-Torricelli problem with weights. 
     We shall investigate three methods to solve them. For details and papers about this topic see:
     https://drive.google.com/folderview?id=0B-LLRo89XGyvVWhhQnROTTVJWlU&usp=sharing
     
     
     
     
     
     +\section{Source Document}
   The following are the authors for "An Efficient Primal-Dual Interior-Point Method for Minimizing a Sum of Euclidean Norms"

    \begin{enumerate}
      \item Knud D.Andersen
      \item Edmund Christiansen
      \item Andrew R. Conn
      \item Michael L. Overton
      
  Abstract. The problem of minimizinga sum of Euclidean norms dates from the 17th century
and may be the earliest example of duality in the mathematical programming literature. This
nonsmooth optimization problem arises in many different kinds of modern scientific applications.
We derive a primal-dual interior-point algorithm for the problem, by applying Newton’s method
directly to a system of nonlinear equations characterizingprimal and dual feasibility and a perturbed
complementarity condition. The main work at each step consists of solvinga system of linear
equations (the Schur complement equations). This Schur complement matrix is not symmetric, unlike
in linear programming. We incorporate a Mehrotra-type predictor-corrector scheme and present
some experimental results comparing several variations of the algorithm, including, as one option,
explicit symmetrization of the Schur complement with a skew corrector term. We also present
results obtained from a code implemented to solve large sparse problems, using a symmetrized Schur
complement. This has been applied to problems arisingin plastic collapse analysis, with hundreds
of thousands of variables and millions of nonzeros in the constraint matrix. The algorithm typically
finds accurate solutions in less than 50 iterations and determines physically meaningful solutions
previously unobtainable.    
      
    \end{enumerate}
    
    
    
    
    




\end{document}
