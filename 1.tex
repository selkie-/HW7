
\documentclass[13pt]{article}

\usepackage[english]{babel}
\usepackage{amsmath}
\usepackage{graphicx}



\title{HW 7}
\author{Xin Guo}

\begin{document}
\maketitle

\begin{AHA}
this homework
\end{AHA}

\section{Introduction}

Create a 2 page LaTeX document that is about your project. Use at least the following features from latex: at least one math formula, lists . Your solution should be a link to a directory on github that contains at least a .tex file and a .pdf file (so I can easily look at the pdf generated from your tex file without having to tex it myself). IMPORTANT: I will only grade this based on the typesetting, not on the mathematical *content* of your document.

\section{Yes \LaTeX{} Yes}
\label{sec:examples}

\subsection{introduction of Knud D. Andersen}
Knud D. Andersen: A Modified Schur-Complement Method for Handling Dense Columns in Interior-Point Methods for Linear Programming. ACM Trans. Math. Softw. 22(3): 348-356 (1996)


\subsection{a book}


Authors: Knud D. Andersen 
Journal: ACM Trans. Math. Softw. Vol. 22 No. 3 Pg. 348-356 [Contents] 
Year: 1996 
Language: English 
Type: journal (article) 
Source: DBLP


\begin{table}
\centering
\begin{tabular}{l|r}
AHA & AHAHA \\\hline
left & yes \\
right & no
\end{tabular}
\caption{\label{tab:widgets}above is a table.}
\end{table}

\subsection{another section}

\letsSee{} what happens 
$$A_n = \frac{M_1 + M_2 + \cdots + M_n}{n}
      = \frac{1}{p}\sum_{i}^{n} M_i$$


\subsection{Last}

members are \dots

\begin{enumerate}
\item Hao
\item Lee
\item Di
\item Xin
\end{enumerate}
\dots 


\end{document}
